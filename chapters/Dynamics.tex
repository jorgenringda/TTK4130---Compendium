\ctitle{Dynamics}
\section{Rigid body kinematics}
\subsection{Rotation Matrices}
Let $\{\Vec{a_1},\Vec{a_2},\Vec{a_3}\}$ and $\{\Vec{b_1},\Vec{b_2},\Vec{b_3}\}$ be the orthogonal bases of two coordinate frames. Then the \textit{coordinate transform} from frame b to frame a is given by
\begin{align}
    \mathbf{R}_b^a = 
    \begin{bmatrix}
    \mathbf{b}_1^a & \mathbf{b}_2^a & \mathbf{b}_3^a
    \end{bmatrix}
\end{align}
That is,
\begin{align}
    \mathbf{v}^a = \mathbf{b}_1^a\mathbf{v}^b
\end{align}.
The $\mathbf{R}_b^a$ can also be thought of simply as a rotation of a vector. Let's pretend forget that the $b$ frame exists, and think only about the frame $a$. Example:
\begin{align}
    \mathbf{b}_1^a=\mathbf{R}_b^a\mathbf{a}_1^a
\end{align}
which shows that the first basis vector of the a frame is rotated to the first basis vector of the b frame, when everything is referred to frame a. In this sense, the matrix represents \textit{a rotation from a to b}. Therefore the rotation matrix $\mathbf{R}_b^a$ is called both a \textit{coordinate transformation from b to a} if we regard it from the first perspective and a \textit{rotation from a to b} if we regard it from the second perspective.

\subsection{Homogeneous transformation matrices}
The homogeneous transformation matrix specifies the position and orientation of a coordinate frame w.r.t. a reference frame. 
\begin{align}
    T_b^a = 
    \begin{bmatrix}
    R_b^a & r_{ab}^a \\
    0 & 1
    \end{bmatrix}
\end{align}
where $r_{ab}^a$ is the origin of frame b in a coordinates. There are three use-cases for the matrix
\begin{enumerate}
    \item Represent a configuration. This means that we can use $T_b^a$ to represent coordinate system b with reference to coordinate system a or represent c from a $T_c^a=T_b^a T_c^b$
    \item Change reference frame of a vector. See note 1 below.
    \item Displace a vector or frame.
\end{enumerate}

\textbf{Note 1:} We cannot write $p_a=T_b^a p_b$ as $T_b^a\in R^{4 \times 4}$ and $p_b \in R^3$. We fix this with the homogeneous coordinate representation of the 3-vector. 
\begin{align}
    \begin{bmatrix}
    p_a \\ 1 
    \end{bmatrix}
    = T_b^a
    \begin{bmatrix}
    p_b \\ 1
    \end{bmatrix}
\end{align}

\subsection{Euler angles}
\subsubsection{Axis-angle representation}
Representing rotation with a rotation axis unit vector $\mathbf{v}$ and a rotation angle $\alpha$. Formula to get from this representation to Rotation matrix is given by
\begin{align}
    R_b^a = \cos(\alpha) I + \sin(\alpha) (\mathbf{v}^a)^x+(1-\cos(\alpha))\mathbf{v}^a(\mathbf{v}^a)^T
\end{align}

The formula to build the skew-symmetric matrix above is:
\begin{align}
    \mathbf{u}^x:=
    \begin{bmatrix}
    0 & -u_3 & u_2 \\
    u_3 & 0 & -u_1 \\
    -u_2 & u_1 & 0
    \end{bmatrix}
\end{align}

\subsubsection{Quaternions}

Unit quaternions are defined by $\eta$ and $\mathbf{\epsilon}$, where 
\begin{align}
    \eta = \cos(\frac{\alpha}{2}) \\
    \mathbf{\theta} = \mathbf{v} \sin(\frac{\alpha}{2})
\end{align}
with $\alpha$ and $\mathbf{\epsilon}$ from the angle-axis representation. To get from quaternions to rotation matrix, the following formula is used. It is based on the Axis-angle representation formula above. 
\begin{align}
    \mathbf{R}_b^a = I + 2\eta\epsilon^\times + 2\epsilon^\times\epsilon^\times
\end{align}

\subsection{Angular velocity}
\begin{align}
    (\boldsymbol{\omega}_{ab}^a)^x=\dot{R_b^a} {R_b^a}^T \\
    \dot{R_b^a}=(\boldsymbol{\omega}_{ab}^a)^x R_b^a \\
    \dot{R_b^a}=R_b^a (\boldsymbol{\omega}_{ab}^b)^x 
\end{align}
If we have 
\begin{align}
    R_b^a = R_1(\phi) R_2(\theta) R_3(\psi)
\end{align}
we derive ordinarily and get
\begin{equation*}
    \begin{bmatrix}
    \dot{\phi} \\
    \dot{\theta} \\
    \dot{\psi}
    \end{bmatrix}
    =
    M^{-1}\boldsymbol{\omega}_{ab}^a
\end{equation*}
Be aware of gimbal lock. With the representation above, the angle in the middle will be causing the gimbal lock for some angle. By doing the derivations and finding the M matrix we can find which angle and what value produces the gimbal lock based on when it becomes rank deficient.

\subsubsection{Differentiation of coordinate vector}
Starting point is
\begin{align}
    \mathbf{u}^a=R_b^a u^b
\end{align}
which we differentiate ordinarily
\begin{align}
    \dot{\mathbf{u}^a} & =\dot{R_b^a}\mathbf{u}^b+R_b^a\dot{\mathbf{u}^b} \\
    & = R_b^a(\boldsymbol{\omega}_{ab}^b \times \mathbf{u}^b + \dot{\mathbf{u}^b})
\end{align}
What we end up with is the formula
\begin{align}
   \prescript{a}{}{\frac{d\Vec{u}}{dt}} = \prescript{b}{}{\frac{d\Vec{u}}{dt}}+ \Vec{\omega}_{ab} \times \Vec{u}
\end{align}

\subsection{Solid Kinematic}
Position of point p
\begin{align}
    \Vec{r}_p=\Vec{r}_0 + \Vec{r}
\end{align}
Velocity of point p
\begin{align}
    \Vec{v}_p & = \prescript{a}{}{\frac{d\Vec{r}_p}{dt}} \\
    & = \Vec{v}_0 + \prescript{b}{}{\frac{d\Vec{r}}{dt}}+\Vec{\omega}\times\Vec{r}
\end{align}
Acceleration of point p
\begin{align}
    \Vec{a}_p & = \prescript{b}{}{\frac{d^2\Vec{r}}{dt^2}}+2\Vec{\omega}_{ab}\times\prescript{b}{}{\frac{d\Vec{r}}{dt}}+\Dot{\Vec{\omega}}_{ab}\times\Vec{r}+\Vec{\omega}_{ab}\times(\Vec{\omega}_{ab}\times\Vec{r})
\end{align}
Explanation of terms:
\begin{enumerate}
    \item $\prescript{b}{}{\frac{d^2\Vec{r}}{dt^2}}$ is acceleration due to p accelerating in b
    \item $2\Vec{\omega}_{ab}\times\prescript{b}{}{\frac{d\Vec{r}}{dt}}$ Coriolis effect
    \item $\Dot{\Vec{\omega}}_{ab}\times\Vec{r}$ is acceleration to to b spinning faster and faster
    \item $\Vec{\omega}_{ab}\times(\Vec{\omega}_{ab}\times\Vec{r})$ centrifugal effects
\end{enumerate}
Simplification: Whenever possible, we attach frame b to solid. This results in $\prescript{b}{}{\frac{d^2\Vec{r}}{dt^2}}=\prescript{b}{}{\frac{d\Vec{r}}{dt}}=0$ which simplifies the formulas above.

\subsection{The center of mass}
The mass of a rigid body b is 
\begin{align}
    m = \int_bdm = \int_b\rho(x,y,z)dV
\end{align}
The center of mass $\Vec{r}_c$ is defined ass
\begin{align}
    \Vec{r}_c = \frac{1}{m}\int_b\Vec{r}_pdm
\end{align}
where $\Vec{r}_p$ is the position of a mass element $dm$ that is fixed in frame b. Note that
\begin{align}
    \int\Vec{r}dm  &= \int\Vec{r}_pdm-\int\Vec{r}_cdm \\
    &= m\Vec{r}_c-m\Vec{r_c} \\
    &=0
\end{align}

\subsection{Other useful formulas}
Relation between linear and angular velocity
\begin{align}
    v = \omega r
\end{align}

\section{Newton-Euler equations of motion}
\subsection{Equations of motion for a rigid body}
\begin{align}
\Vec{F}_{bc} = m\Vec{a}_c \\
\Vec{T}_{bc} = \Vec{M}_{b/c}*\Dot{\Vec{\omega}}_{ib}+\Vec{\omega}_{ib}\times(\Vec{M}_{b/c}\Vec{\omega}_{ib})
\end{align}

\subsection{Kinetic energy}

\begin{align}
    \mathcal{T} = \frac{1}{2} m (\mathbf{v}_c^b)^\top \mathbf{v}_c^b +  \frac{1}{2} (\boldsymbol{\omega}_{ib}^b)^\top \mathbf{M}_{b/c}^b \boldsymbol{\omega}_{ib}^b
\end{align}
Subscript c denotes center of mass and superscript b denotes a coordinate vector/matrix in frame b. $\omega_{ib}^b$ is the angular velocity of frame b relative to frame i. $\mathbf{M}_{b/c}^b$ is the inertia matrix of b about c given in b. I.e. the inertia matrix of the rigid body about the center of mass

\subsection{Inertia matrix}
\begin{equation}
    \mathbf{M}_{b/c}^b = -\int_b (\mathbf{r}^b)^\times (\mathbf{r}^b)^\times dm = \int_b \left[ (\mathbf{r}^b)^\top \mathbf{r}^b \mathbf{I} - \mathbf{r}^b (\mathbf{r}^b)^\top \right] dm
\end{equation}
Note that $\mathbf{M}_{b/c}^b$ is positive definite since the kinteic energy $\mathcal{T}\geq0$. Note also that the integral above is a triple integral of a $3\times3-matrix$. About a specified axis the formula reduces to 

\begin{equation}
    I = \int_{b} (\mathbf{r}^b)^\top \mathbf{r}^b dm
\end{equation}

\subsection{Parallel axis theorem}
The inertia matrix of b about a point o is given by 
\begin{equation}
    \mathbf{M}_{b/o}^b = \mathbf{M}_{b/c}^b - m (\mathbf{r}_g^b)^\times (\mathbf{r}_g^b)^\times =
    \mathbf{M}_{b/c}^b + m\left[ (\mathbf{r}_g^b)^\top \mathbf{r}_g^b \mathbf{I} - \mathbf{r}_g^b (\mathbf{r}_g^b)^\top \right]
\end{equation}
where $\mathbf{r}_g^b$ is the vector from the point o to the center of mass c. If o is the origin this corresponds to $\mathbf{r}_c^b$. In it's simplest form with two parallel axes, the formula reduces to 
\begin{align}
    I = I_c + md^2
\end{align}
where $I_c$ is the moment of inertia about the axis through the center of mass and d is the distance between the axes. 

\textbf{Implicit Function Theorem}, Jacobian $\frac{\partial \varphi(\overline{\mathrm{x}}, \overline{\mathrm{y}})}{\partial \mathrm{x}}$ being full rank is required. ++


\section{Lagrange Mechanics}
\textit{Modeling mechanical systems using an "energy-based" approach}
\subsection{Kinetic energy}
\begin{align}
    \mathcal{T} = \frac{1}{2}\Dot{q}^TW(q)\Dot{q}
\end{align}
where
\begin{align}
    W(q) = \sum_{i=1}^Nm_i\frac{\partial p_i}{\partial q}^T\frac{\partial p_i}{\partial q}
\end{align}

\subsection{Potential Energy}
Comes in many forms, gravity is most commonly seen in this course
\begin{align}
    \mathcal{V} = mgz
\end{align}
where z is height in field. We have also seen potential energy in spring and combination.

\subsection{Lagrange equation}
The Lagrange function
\begin{align}
    \mathcal{L}(q,\Dot{q})=\mathcal{T}(q,\Dot{q})-V(q)-z^Tc(q)
\end{align}
is used in the Lagrange equation
\begin{align}
    \frac{d}{dt}\frac{\partial\mathcal{L}}{\partial \dot{q}}^T-\frac{\partial\mathcal{L}}{\partial q}^T=Q
\end{align}
where $c(q)$ are constraint equations, z are Lagrange multipliers (the forces that hold the constraints in place) and Q is generalized forces we get by using 
\begin{align}
    \sum_{i=1}^N\frac{\partial p_i}{\partial q}^T F_i
\end{align}

Examples: \\
$q = x \in \mathbb{R}^1$, 
$q = \begin{bmatrix} x\\ \theta \end{bmatrix} \in \mathbb{R}^2$, 
$q = \begin{bmatrix} x\\ y\\ z \end{bmatrix} \in \mathbb{R}^3$. \\
The purpose of the Lagrange equation is to derive model for system dynamics. Handy remarks:
\begin{itemize}
    \item[$\diamond$] As long as the objects we model with regards to have mass, W(q) is always full rank.
    \item[$\diamond$] Model complexity due to W(q) not constant.
    \item[$\diamond$] Constrained Lagrange allows for use of Cartesian coordinates as general coordinates.
    \item[$\diamond$] W will always be constant when a collection of Cartesian coordinates are used as q
\end{itemize}




